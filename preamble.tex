%%%%%%%%%%%%%%%%%%%%%%%%%%%%%%%%%%%%%%%%%%%%%%%%%%%%%%%%%%%%%%%%%%%%%%%%%%%%%
% preamble.tex
%%%%%%%%%%%%%%%%%%%%%%%%%%%%%%%%%%%%%%%%%%%%%%%%%%%%%%%%%%%%%%%%%%%%%%%%%%%%%

\usepackage{lmodern}
\renewcommand{\sfdefault}{lmss}
\renewcommand{\ttdefault}{lmtt}

\usepackage[T1]{fontenc}
\usepackage[utf8]{inputenc}

\usepackage{geometry}
\geometry{verbose,tmargin=40mm,bmargin=46mm,lmargin=38mm,rmargin=32mm}

\setcounter{secnumdepth}{3}
\setcounter{tocdepth}{2}

\usepackage{color}
\usepackage{babel}
\usepackage{longtable}
\usepackage{float}
\usepackage{mathtools}
\usepackage{enumitem}
\usepackage{amsmath}
\usepackage{amsthm}
\usepackage{amssymb}
\usepackage{mathdots}
\usepackage{stmaryrd}

%%%%%%%%%%%%%%%%%%%%%%%%%%%%%%%%%%%%%%%%%%%%%%%%%%%%%%%%%%%%%%%%%%%%%%%%%%%%%

\usepackage[unicode=true]{hyperref}

\hypersetup{
  pdftitle           = {\licTitle},
  pdfauthor          = {\licAuthor},
  pdfsubject         = {Licentiate Thesis in Theoretical Physics},
  bookmarks          = true,
  bookmarksnumbered  = true,
  bookmarksopen      = true,
  bookmarksopenlevel = 1,
  breaklinks         = true,
  pdfborder          = {0 0 0},
  pdfborderstyle     = {},
  backref            = false,
  colorlinks         = true,
  linkcolor          = blue,
  urlcolor           = blue,
  citecolor          = blue,
  anchorcolor        = green,
  pdfstartview       = {Fit},
  pdfpagelayout      = {TwoPageRight},
  linktocpage, pagebackref, hyperfigures
}

% Enable hyperlinks for S5 paper too
%\ifSpaper
%  \hypersetup{colorlinks=false}
%\fi

%%%%%%%%%%%%%%%%%%%%%%%%%%%%%%%%%%%%%%%%%%%%%%%%%%%%%%%%%%%%%%%%%%%%%%%%%%%%%

% S5 papersize = { 165mm, 242mm } * 12 / 10.95 + { 6mm, 0mm }

\ifSpaper
  \geometry{
    verbose,
    papersize = { 184.82mm, 265.2mm }, 
    tmargin   = 27.1mm,
    bmargin   = 27.1mm,
    lmargin   = 22.41mm,
    rmargin   = 22.41mm,
    marginparwidth = 20mm
  }
\fi

%%%%%%%%%%%%%%%%%%%%%%%%%%%%%%%%%%%%%%%%%%%%%%%%%%%%%%%%%%%%%%%%%%%%%%%%%%%%%

% Optionally show the diagram of the page layout

\ifShowLayout
  \usepackage[noframe]{showframe} 
  \usepackage{layouts}
  \makeatletter
  \let\FontSize=\f@size
  \makeatother
\fi

%%%%%%%%%%%%%%%%%%%%%%%%%%%%%%%%%%%%%%%%%%%%%%%%%%%%%%%%%%%%%%%%%%%%%%%%%%%%%

% Use sanserif for headings

\usepackage[{sf,bf}]{titlesec}

%%%%%%%%%%%%%%%%%%%%%%%%%%%%%%%%%%%%%%%%%%%%%%%%%%%%%%%%%%%%%%%%%%%%%%%%%%%%%

% \usepackage[margin=10pt,font=small,labelfont=bf]{caption} 
\usepackage[margin=10pt,labelfont=bf]{caption} 

% Turn off nasty worning for overfull and underfull hboxes
% \hbadness=10000
% \hfuzz=50pt

%%%%%%%%%%%%%%%%%%%%%%%%%%%%%%%%%%%%%%%%%%%%%%%%%%%%%%%%%%%%%%%%%%%%%%%%%%%%%

\usepackage{cite} % Compressed citations

%%%%%%%%%%%%%%%%%%%%%%%%%%%%%%%%%%%%%%%%%%%%%%%%%%%%%%%%%%%%%%%%%%%%%%%%%%%%%

\pdfmapfile{+rsfso.map}  % More upright mathscr
\DeclareSymbolFont{rsfso}{U}{rsfso}{m}{n}
\DeclareSymbolFontAlphabet{\mathscr}{rsfso}

% \usepackage[scaled=1.1]{rsfso}   % Turn also mathcal into rfso

%%%%%%%%%%%%%%%%%%%%%%%%%%%%%%%%%%%%%%%%%%%%%%%%%%%%%%%%%%%%%%%%%%%%%%%%%%%%%

%\usepackage{etoolbox} % a toolbox of programming tools

\usepackage{xfrac}  % Split-level fractions (\sfrac)

%%%%%%%%%%%%%%%%%%%%%%%%%%%%%%%%%%%%%%%%%%%%%%%%%%%%%%%%%%%%%%%%%%%%%%%%%%%%%

%\usepackage{graphicx}
%\DeclareGraphicsExtensions{.png, .jpg, .jpeg, .pdf} %GIF doesn't work
%\pdfcompresslevel=9
%\graphicspath{{figures/}}

%%%%%%%%%%%%%%%%%%%%%%%%%%%%%%%%%%%%%%%%%%%%%%%%%%%%%%%%%%%%%%%%%%%%%%%%%%%%%

\usepackage{tensind} 
\tensordelimiter{?}

%%%%%%%%%%%%%%%%%%%%%%%%%%%%%%%%%%%%%%%%%%%%%%%%%%%%%%%%%%%%%%%%%%%%%%%%%%%%%

\usepackage{xcolor}
\usepackage{colortbl}

\let\myHlineC\hline

\newcommand{\bHlineC}{
  \renewcommand{\hline}{\arrayrulecolor{lightgray}\myHlineC\arrayrulecolor{black}}
  \newcolumntype{|}{!{\color{lightgray}\vline}}
}

\newcommand{\eHlineC}{
  \renewcommand{\hline}{\myHlineC}
  \newcolumntype{|}{!{\color{black}\vline}}
}

%%%%%%%%%%%%%%%%%%%%%%%%%%%%%%%%%%%%%%%%%%%%%%%%%%%%%%%%%%%%%%%%%%%%%%%%%%%%%

\newcommand{\bSe}{\begin{subequations}} 
\newcommand{\eSe}{\end{subequations}}

\newcommand{\bWe}{\begin{widetext}} 
\newcommand{\eWe}{\end{widetext}}

%%%%%%%%%%%%%%%%%%%%%%%%%%%%%%%%%%%%%%%%%%%%%%%%%%%%%%%%%%%%%%%%%%%%%%%%%%%%%

\allowdisplaybreaks

%%%%%%%%%%%%%%%%%%%%%%%%%%%%%%%%%%%%%%%%%%%%%%%%%%%%%%%%%%%%%%%%%%%%%%%%%%%%%

% Remove if we do not have tikz

\usepackage{tikz,amsmath,amssymb,bm,color,cancel}
\usetikzlibrary{shapes,arrows}
\usetikzlibrary{calc}
\usetikzlibrary{positioning}
\usetikzlibrary{decorations.pathreplacing}

%%%%%%%%%%%%%%%%%%%%%%%%%%%%%%%%%%%%%%%%%%%%%%%%%%%%%%%%%%%%%%%%%%%%%%%%%%%%%

\usepackage{lastpage}

%%%%%%%%%%%%%%%%%%%%%%%%%%%%%%%%%%%%%%%%%%%%%%%%%%%%%%%%%%%%%%%%%%%%%%%%%%%%%

% Load fancyhdr after specifying the geometry (bogus if loaded before)
\usepackage{fancyhdr}
\usepackage[iso]{datetime}

\fancypagestyle{empty}
{
    \fancyhf{} % clear all header and footer fields
}

\fancypagestyle{plain}
{
    \fancyhf{} % clear all header and footer fields
    \fancyfoot[LE,RO]{\thepage}
    \fancyfoot[RE,LO]{}
    \renewcommand{\headrulewidth}{0pt}
    \renewcommand{\footrulewidth}{0pt}
}

\fancyhf{} % clear all header and footer fields

\fancyhead[LE,RO]{\thepage}
\fancyhead[RE]{\small\leftmark}
\fancyhead[LO]{\small\rightmark}
\fancyhead[CE]{}

\renewcommand{\headrulewidth}{0pt}
\renewcommand{\footrulewidth}{0pt}

%%%%%%%%%%%%%%%%%%%%%%%%%%%%%%%%%%%%%%%%%%%%%%%%%%%%%%%%%%%%%%%%%%%%%%%%%%%%%

\pagestyle{fancy}

% Don't convert marks to upperase

% printindex sets the pagestyle to plain, to counter this, we can redefine this pagestyle. 
% If we use the fancy pagestyle throughout our document add the following to the preamble:
%\renewcommand{\ps@plain}{\pagestyle{fancy}}

\frontmatter

%%%%%%%%%%%%%%%%%%%%%%%%%%%%%%%%%%%%%%%%%%%%%%%%%%%%%%%%%%%%%%%%%%%%%%%%%%%%%

\usepackage[titletoc]{appendix}

\usepackage[nottoc,notlof,notlot]{tocbibind}

\renewcommand{\sectionmark}[1]{ \markright{\thesection\ #1} }
\renewcommand{\chaptermark}[1]{ \markboth{\chaptername\ \thechapter.\ #1}{\chaptername\ \thechapter.\ #1} }
\renewcommand{\tocetcmark}[1]{ \markboth{#1}{#1} }

% clear empty double pages

%\let\origdoublepage\cleardoublepage
%\newcommand{\clearemptydoublepage}{%
%  \clearpage
%  {\pagestyle{empty}\origdoublepage}%
%}
%\let\cleardoublepage\clearemptydoublepage

\usepackage{emptypage}

%%%%%%%%%%%%%%%%%%%%%%%%%%%%%%%%%%%%%%%%%%%%%%%%%%%%%%%%%%%%%%%%%%%%%%%%%%%%%

\numberwithin{equation}{chapter}
\renewcommand{\theequation}{\thechapter.\arabic{equation}}

\numberwithin{figure}{chapter}
\renewcommand{\thefigure}{\thechapter.\arabic{figure}}

\numberwithin{table}{chapter}
\renewcommand{\thetable}{\thechapter.\arabic{table}}

%%%%%%%%%%%%%%%%%%%%%%%%%%%%%%%%%%%%%%%%%%%%%%%%%%%%%%%%%%%%%%%%%%%%%%%%%%%%%

% \footnotesep is the space between footnotes:
\setlength{\footnotesep}{4mm}

% \footins is the space between the text body and the footnotes:
\setlength{\skip\footins}{6mm}

\selectlanguage{english}

%%%%%%%%%%%%%%%%%%%%%%%%%%%%%%%%%%%%%%%%%%%%%%%%%%%%%%%%%%%%%%%%%%%%%%%%%%%%%

% To include papers:

\usepackage{pdfpages}
\usepackage{ifoddpage}

%%%%%%%%%%%%%%%%%%%%%%%%%%%%%%%%%%%%%%%%%%%%%%%%%%%%%%%%%%%%%%%%%%%%%%%%%%%%%

\usepackage{tabularx}

\newenvironment{pmatrixc}{
  \bgroup\renewcommand{\arraystretch}{1}\begin{pmatrix}
}{
  \end{pmatrix}\egroup
}

\newenvironment{pmatrixr}{
  \bgroup\renewcommand{\arraystretch}{1}\begin{pmatrix*}[r]
}{
  \end{pmatrix*}\egroup
}

%%%%%%%%%%%%%%%%%%%%%%%%%%%%%%%%%%%%%%%%%%%%%%%%%%%%%%%%%%%%%%%%%%%%%%%%%%%%%

\newlength{\InnerEdge}
\newlength{\OuterEdge}
\newcommand{\FrameEdgeColor}{blue!30}
\newcommand{\ShowSubGridColor}{gray!15}
\newcommand{\ShowGridColor}{gray!25}
\newcommand{\ShowGridTextColor}{gray!50}

\ifSpaper
  \setlength{\InnerEdge}{6mm}
\else
  \setlength{\InnerEdge}{8mm}
\fi

\ifSpaper
   \setlength{\OuterEdge}{6mm}
\else
   \setlength{\OuterEdge}{6mm}
\fi

\newcommand{\MarkFrameEdges}{%   
  \begin{tikzpicture}[x=1mm, y=1mm, remember picture, overlay]
      \checkoddpage\ifoddpage
        \draw [dashed,\FrameEdgeColor] 
          ($(current page.south west) + (\InnerEdge,0)$) -- 
          ($(current page.south west) + (\InnerEdge,\paperheight)$);
        \draw [dashed,\FrameEdgeColor] 
          ($(current page.south east) + (-\OuterEdge,0)$) -- 
          ($(current page.south east) + (-\OuterEdge,\paperheight)$);
      \else
        \draw [dashed,\FrameEdgeColor] 
          ($(current page.south east) + (-\InnerEdge,0)$) -- 
          ($(current page.south east) + (-\InnerEdge,\paperheight)$);
        \draw [dashed,\FrameEdgeColor] 
          ($(current page.south west) + (\OuterEdge,0)$) -- 
          ($(current page.south west) + (\OuterEdge,\paperheight)$);
      \fi
  \end{tikzpicture}
}

\newcommand{\ShowGrid}{%   
  \begin{tikzpicture}[x=1mm, y=1mm, remember picture, overlay]
      \checkoddpage\ifoddpage
        \draw[step=1,\ShowSubGridColor,very thin] 
            ($(current page.south west) + (0,0)$) grid 
            ($(current page.south west) + (\paperwidth,\paperheight)$);
        \draw[step=10,\ShowGridColor,very thin] 
            ($(current page.south west) + (0,0)$) grid 
            ($(current page.south west) + (\paperwidth,\paperheight)$);
        \foreach \i in {1,...,30} {
            \node [anchor=west,align=right,\ShowGridTextColor] 
            at ($(current page.south west) + (1,\i * 10)$) {$\i$};
        }
        \foreach \i in {1,...,21} {
            \node [anchor=south,align=center,\ShowGridTextColor] 
            at ($(current page.south west) + (\i * 10,1)$) {$\i$};
        }
        \draw [thin,red] (current page.south west)
           -- ($(current page.south west) + (10,10)$);
      \else
        \draw[step=1,\ShowSubGridColor,very thin] 
            ($(current page.south east) + (0,0)$) grid 
            ($(current page.south east) + (-\paperwidth,\paperheight)$);
        \draw[step=10,\ShowGridColor,very thin] 
            ($(current page.south east) + (0,0)$) grid 
            ($(current page.south east) + (-\paperwidth,\paperheight)$);
        \foreach \i in {1,...,30} {
            \node [anchor=east,align=right,\ShowGridTextColor] 
            at ($(current page.south east) + (-1,\i * 10)$) {$\i$};
        }
        \foreach \i in {1,...,21} {
            \node [anchor=south,align=center,\ShowGridTextColor] 
            at ($(current page.south east) + (-\i * 10,1)$) {$\i$};
        }
        \draw [thin,red] (current page.south east)
           -- ($(current page.south east) + (-10,10)$);
      \fi
  \end{tikzpicture}
}

\ifx\ShowFrameColor\unknown\else
  \renewcommand*\ShowFrameColor{\color{red!10}}
  \AddToShipoutPictureBG{\ShowFramePicture} 
  \ifShowGrid
    \AddToShipoutPictureBG{\ShowGrid}
  \fi
  \AddToShipoutPictureFG{\MarkFrameEdges}
\fi

%%%%%%%%%%%%%%%%%%%%%%%%%%%%%%%%%%%%%%%%%%%%%%%%%%%%%%%%%%%%%%%%%%%%%%%%%%%%%

\makeatletter

\newlength{\TopOffset}
\newlength{\ThumbBoxH}
\newlength{\ThumbBoxW}
\newlength{\ThumbBoxX}
\newlength{\ThumbBoxLargeX}
\newlength{\ThumbBoxY}
\newlength{\OuterMargin}

\newcommand{\ThumbBoxColor}{black!30}

\setlength{\TopOffset}{25.4mm+\hoffset+\topmargin+\headheight+\headsep}
\setlength{\ThumbBoxH}{23mm}
\setlength{\ThumbBoxW}{70mm}
\setlength{\ThumbBoxX}{14mm}

\ifSpaper
   \setlength{\ThumbBoxLargeX}{30mm}
\else
   \setlength{\ThumbBoxLargeX}{24mm}
\fi

\setlength{\ThumbBoxY}{\TopOffset-\ThumbBoxH/2}

\setlength{\OuterMargin}{25.4mm+\hoffset+\evensidemargin}

\newcommand{\ThumbBox}[2]{%
  \begin{tikzpicture}[x=1mm, y=1mm, remember picture, overlay]
      \checkoddpage\ifoddpage
        \node [anchor=north,rotate=90,draw,rectangle, very thin, rounded corners=5pt,
          color=\ThumbBoxColor,fill=\ThumbBoxColor,minimum width=\ThumbBoxH,minimum height=\ThumbBoxW] 
          at ($(current page.north east) - (\ThumbBoxX,\ThumbBoxY+\value{chapter}*\ThumbBoxH+#2*\ThumbBoxH)$) 
          {};
        \node [anchor=north,rotate=90,text width=\ThumbBoxH,align=flush center,font=\sffamily] 
          at ($(current page.north east) - (\ThumbBoxX-1,\ThumbBoxY+\value{chapter}*\ThumbBoxH+#2*\ThumbBoxH)$) 
          {\color{white}\textbf{#1}};
      \else
        \node [anchor=south,rotate=90,draw,rectangle, very thin, rounded corners=5pt,
          color=\ThumbBoxColor,fill=\ThumbBoxColor,minimum width=\ThumbBoxH,minimum height=\ThumbBoxW] 
          at ($(current page.north west) - (-\ThumbBoxX,\ThumbBoxY+\value{chapter}*\ThumbBoxH+#2*\ThumbBoxH)$) 
          {};
        \node [anchor=south,rotate=90,text width=\ThumbBoxH,align=flush center,font=\sffamily] 
          at ($(current page.north west) - (-\ThumbBoxX+1,\ThumbBoxY+\value{chapter}*\ThumbBoxH+#2*\ThumbBoxH)$) 
          {\color{white}\textbf{#1}};
      \fi
  \end{tikzpicture}
}

\newcommand{\ThumbBoxLarge}[2]{%
  \checkoddpage\ifoddpage
    \begin{tikzpicture}[x=1mm, y=1mm, remember picture, overlay]
      \node [anchor = west,
        draw, rectangle, very thin, rounded corners=5pt, inner sep = 5,
        fill = \ThumbBoxColor, color = \ThumbBoxColor,
        minimum height = \ThumbBoxH, minimum width = \ThumbBoxW, text width = \ThumbBoxW,
        align = flush left, font = \sffamily] 
        at ($(current page.north east) - (\OuterMargin+\ThumbBoxLargeX,\ThumbBoxY+\value{chapter}*\ThumbBoxH+#2*\ThumbBoxH)$)
        {\color{white}\fontsize{23pt}{0em}\selectfont\textbf{~~#1}};
    \end{tikzpicture}
  \fi
}

\newcommand{\labelPaper}[1]{%
   \let\orglabel\label
   \let\label\@gobble
   \edef\@currentlabel{#1\unskip}%
   \let\label\orglabel
}

\newcommand{\overlayPaperInfo}[1]{%
  \begin{tikzpicture}[x=1mm, y=1mm, remember picture, overlay]
    \node [anchor=south east,align=flush left,text width=\textwidth, inner sep = 0] 
      at ($(current page.north east) - (\OuterMargin,\TopOffset+\textheight)$) {#1};
  \end{tikzpicture}
}

\newcounter{PaperSubFolio}
\newcommand{\PaperSubFolioFontSize}{12}

\ifShowLayout
  \tikzset{folioFill/.style={red!10}}
\else
  \tikzset{folioFill/.style={white,fill=white}}
\fi

\newcommand{\overlayPaperFolio}[3]{%
  \begin{tikzpicture}[x=1mm, y=1mm, remember picture, overlay]
      \checkoddpage\ifoddpage
        \node [anchor=south,draw,rectangle,folioFill] 
          at ($(current page.south) + (#2,#3)$)
          {\fontsize{\PaperSubFolioFontSize}{0}\selectfont\color{black}
          \textsc{Paper \PaperID}~--~\thePaperSubFolio~\color{black!60}(\thepage)};
      \else
        \node [anchor=south,draw,rectangle,folioFill] 
          at ($(current page.south) + (-#2,#3)$)
          {\fontsize{\PaperSubFolioFontSize}{0}\selectfont\color{black}
          \textsc{Paper \PaperID}~--~\thePaperSubFolio~\color{black!60}(\thepage)};
      \fi
  \end{tikzpicture}
  \stepcounter{PaperSubFolio}
}

\makeatother

%%%%%%%%%%%%%%%%%%%%%%%%%%%%%%%%%%%%%%%%%%%%%%%%%%%%%%%%%%%%%%%%%%%%%%%%%%%%%

\makeatletter
\newcommand{\lofwithouttitle}{\@starttoc{lof}}
\newcommand{\lotwithouttitle}{\@starttoc{lot}}
\newcommand{\tocwithouttitle}{\@starttoc{toc}}
\makeatother

%%%%%%%%%%%%%%%%%%%%%%%%%%%%%%%%%%%%%%%%%%%%%%%%%%%%%%%%%%%%%%%%%%%%%%%%%%%%%

\usepackage{tocloft}

\cftpagenumbersoff{part}

\renewcommand{\cftpartfont}{\scshape\normalsize}
\renewcommand{\cftpartpagefont}{\bfseries\rmfamily\normalsize}
\renewcommand{\cftbeforepartskip}{6mm}
\renewcommand{\cftpartafterpnum}{\vspace{2mm}}

\renewcommand{\cftchapfont}{\rmfamily\normalsize}
\renewcommand{\cftchappagefont}{\rmfamily\normalsize}
\renewcommand{\cftchapleader}{\cftdotfill{\cftchapdotsep}}
\renewcommand{\cftchapdotsep}{\cftdotsep}
\setlength{\cftchapindent}{8mm}
\setlength{\cftbeforechapskip}{1mm}

\renewcommand{\cftsecfont}{\rmfamily\normalsize}
\renewcommand{\cftsecpagefont}{\rmfamily\normalsize}
\setlength{\cftsecindent}{14mm}

\setlength{\cftsubsecindent}{19mm}
\setlength{\cftsubsubsecindent}{26mm}

\setlength{\cftbeforefigskip}{\cftbeforechapskip}
\setlength{\cftbeforetabskip}{\cftbeforechapskip}

%%%%%%%%%%%%%%%%%%%%%%%%%%%%%%%%%%%%%%%%%%%%%%%%%%%%%%%%%%%%%%%%%%%%%%%%%%%%%
